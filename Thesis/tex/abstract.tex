%# -*- coding: utf-8-unix -*-
%%==================================================
%% abstract.tex for SJTU Master Thesis
%%==================================================

\begin{abstract}

图像主观质量评价是人对其通过视觉感知的图像质量在生理和心理上所给出的偏好性的反应。作为一种新兴的媒体,立体图像(Stereoscopic 3D(S3D))凭借感知重建场景深度这一优势,迅速地在家庭、电影院以及其他娱乐产业等领域得到普及。由于平面格式的立体图像/视频在观看过程中因汇聚-调节矛盾容易导致不舒适、视觉疲劳等问题,立体图像的质量评价就显得尤为重要。近年来,随着生理学的发展,人们对人类视觉系统的理解逐步加深,越来越多的研究开始关注人眼视觉特征在图像质量评价方面的作用,一系列很有创造性的基于人类视觉特征的模型被应用到了立体图像质量评价中。眼动数据作为人眼观看图像/视频的直接记录,开始广泛地应用于基于人类视觉特征的模型研究,很好地改善了已有的基于整体图像特征的评价模型效果。但是,这些方法只利用了眼动数据注视区域这一单一特征,而忽略了眼动数据所携带的其他有价值的特征,如注视点的跳变,辐辏调节等。因此,本文的主要目的是对眼动数据做特征性分析,找出与图像质量相关的眼动数据特征,并将其应用在立体图像质量评价当中。

针对这一问题,首先创建了立体图像-眼动数据库(Stereoscopic Image-Eyetracking Data(SIED))。SIED的创建包括了三部分:图像库的建立,眼动数据采集系统开发,眼动实验的设计与实施。SIED的图像是由包括不同场景、不同深度的11幅图像进行了7种出入屏的调整得到的7*11幅图像构成。眼动数据的采集是基于Tobii 眼动仪SDK开发的在线眼动数据采集系统来完成。在线眼动数据采集系统与在线的网络质量评价平台\footnote{\url{http://3dvqa.sjtu.edu.cn/3dvqa}}通信,获取在线发布的测试任务,采集单刺激模式下被试者测试过程的眼动数据,上传主观评价结果及眼动数据。在眼动数据采集过程中,共设计了三个实验,立体视敏度检验实验测试了被试者立体感是否正常;3D校正实验用来获取被试者立体感的系统性偏差,并据此进行眼动数据校正;立体图像的眼动实验采集被试者观看图像时的眼动数据,据此来建立基于眼动数据的立体图像质量评价模型。

为了更好地利用眼动数据,本文对眼动数据的处理方法进行了研究,在眼动数据的滤波方面,改进了现有的基于双目视点平均位置滤波的2D场景的算法,提出了基于辅眼视点的3D滤波算法。对于眼动数据的3D校正,针对眼动数据采集配置方式的不同,改进了现有的基于视差偏差的校正算法,提出了基于视差角偏差的3D校正算法。同时本文提出了一种立体图像眼动数据可视化的方法——立体图像分层注视密度图法,该方法建立在基于视差图分割的3D图像的分层立体表示方法之上,并在此基础上根据眼动数据的视差角确定注视区域的深度层次,分层创建注视密度图,形成眼动数据的立体可视化表示。其结果可以很好地表示出立体场景下人眼注视的热点区域及其深度。其分层的思想也在后续的特征提取中发挥了重要的作用。

目前还没有基于眼动数据特征的立体图像质量评价模型。针对这一研究现状,本文在对眼动数据处理的基础上,分析研究了眼动数据中所包含的各类特征,主要包括:静态特征、动态特征以及视差角特征。静态特征反映了眼睛的注视结果,用来描述眼睛注视区域的大小,不同区域注视的时长等信息。在此提取的特征包括注视点得个数、注视平均时长等;动态特征主要反映了眼睛的运动过程,涉及多个显著性区域间的跳变,眼睛的来回扫视等视觉过程。其特征可以用扫视幅度,扫视次数等来描述;视差角反映了眼睛注视的深度信息以及深度变化信息,与人眼在不同景深区域的调节相关,其特征主要包括了视差角的均值、方差、深度调节幅度等。然后利用SVR回归模型建立了基于眼动数据的立体图像质量模型。模型结果表明,MOS值与眼动数据的特征有很大的关联性,特别与眼动数据的视差角特征的关联更加明显,其与视差角均值的相关度达到0.72(线性相关系数)以上,与深度调节的幅度的关联也达到了0.4以上。因此,利用眼动数据来做立体图像质量评价是可行的。

\keywords{\large 立体图像质量评价 \quad 立体注视真值标注\quad 眼动数据 \quad 立体视觉眼动特征}
\end{abstract}

\begin{englishabstract}

Image Quality Assessment(IQA) is one's reaction for an image that comes into his/her view, which is determined by human physiological and psychological feelings. The image quality plays a key roles in information transmitting that the image carried. As a new medium,  Stereoscopic images (Stereoscopic/3D (S3D)) has an advantage that it can make one feel in the real scenario, which made it quickly spread in household,cinemas and the entertainment industry. Since the stereoscopic image/video has a problem that it makes one feel discomfort and visual fatigue, the quality assessment of it is very important.  Traditional methods of stereo image quality assessment is mainly concentrated in 2D image quality assessment methods applying, 2D image quality assessment plus 3D features such as disparity. In recent years,  The breakthroughs of the human visual system have gradually deepened with the development of physiology. More and more research has been focused on the performance of human visual features in Image Quality Assessment. A series of models  based on human visual features have been creatively applied to assess the quality of the stereo image. In these applications,  eyetracking data is widely used in the model based on human visual features, as a direct record of the movement of human eyes when the eyes are watching images and videos. However, most of the models use the eyetracking data to detect the saliency map of an image and then use the feature of the salient area to replace the feature of whole image. This method can obviously improve the performance of the models that based on the feature of whole image. However, this method only uses the fixation regions labeled by eyetracking data, ignoring other valuable features of  eyetracking data. In our thesis, we aimed at find features of eyetracking data which can be used to the assessment of Stereo images.

To solve this problem, a stereoscopic image database(Stereoscopic Image-Eyetracking Data(SIED)) was created. SIED is composed of 7*11 images. These images are selected from different scenarios and have different depths range. We also adjust their disparity to move objects in images to different depth planes which can create various image quality。meanwhile, the database is useful when we make comparative analysis of eyetracking features that are obtained from different disparity distribution in the same scenario. The collection of eyetracking data is applied the system which is developed by ourselves based Tobii SDK and Qt. The system can collect the eyetracking data  in different models such as test with task or without task. This system was also expanded to some developments specially such as task published online and offline,and the data stored in servers or local,this makes the system applied more conveniently.  In the process of data collection,  three experiments were designed. first, we valid every participant's stereoacurity to make sure they have the normal stereo perception. secondly,  a 3D calibration is applied to get the bias in 3D perception for everyone,  which is used to calibrate the eyetracking data in our main experiment. finally, an experiment that collect eyetraking data under 3D image was shown was designed.The SIED contains 77 stereoscopic images(and their disparity maps), the result of stereoacurity test for all participant and all eyetracking data related to 3D calibration and stereo image.

In order to make use of eyetracking data effectively,the processing methods of eyetracking data were studied,too. We improved the filter and calibration algorithms in 2D regions to fit the 3D fact. At the same time,  a representation of stereoscopic image based on disparity map segmented is proposed, which is the skeleton of our stereo fixation density maps who can show not only the area we fixated,but also the real depth our fixation was.  Some analysis between human visual system and eye movement were done based on stereo fixation density maps,too.

In the end, the eyetracking data features was proposed which contains static features(about fixation area ), dynamic features(related to saccade) and disparity features of eyetracking data,then the SVR regression method was applied to obtain the relevancy between eyetracking data features and MOS values. The results showed that the features of the eyetracking data is related to the MOS values,especially in the disparity features.

\englishkeywords{\large Stereoscopic Image Quality Assessment, Stereoscopic Fixation Density Map, Eyetracking Data, Stereo Eye Movement Features}
\end{englishabstract}

